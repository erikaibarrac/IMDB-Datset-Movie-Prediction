\documentclass{article}
\usepackage{amsmath}
\usepackage{biblatex}

\addbibresource{references.bib}
\newcommand{\adjustmentyear}{2020}

\title{Future work}

\begin{document}
    \maketitle

    Now that the data is cleaned, we will attempt to predict what factors will go into making the best possible movie with a specified budget and set number of prominent actors, using the two metrics: gross income, and metascore.
    Gross income is the worldwide income that the movie brings in.
    Metascore is the score, out of 100, given to a movie by critics and is used in considerations for academy awards.

    To do this, we will perform three regressions using the data to build three models.
    The first model will be used to predict gross income.
    The second model will be used to predict metascore and will use the ``filled'' metascore values that were estimated from votes and average vote.
    The third model will also be used to predict metascore, but will use the small dataset that contains real metascore values.
    Using the coefficients on these models, we will determine which actors, directors, writers, and production companies will give the movie that will maximize the chosen metric.
\end{document}